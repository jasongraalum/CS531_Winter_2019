\documentclass[11pt]{article}
\usepackage{amsmath,amssymb,amsthm}
\usepackage{parskip}
%\usepackage[margin=1in]{geometry}
\newcommand{\encode}[1]{\langle #1 \rangle}

\let\tab\quad

\title{\bf Assignment 1 \\[2ex]
\rm\normalsize CS 531 \\ Due January 16, 2019}
\date{January 13, 20190}
\author{Jason Graalum}

\begin{document}
\maketitle

\begin{itemize}
\item Read the Class Syllabus (the class webpage is cs.pdx.edu/~karavan/perf ).
\item We will follow these steps in the Linux Lab during our class period:
\item Read and Step through the examples in the Introduction to Gprof link from the class Lectures page
\item Read the gprof paper (the 2004 version):\\
Susan L. Graham, Peter B. Kessler, and Marshall K. McKusick. 2004. gprof: a call graph execution profiler. SIGPLAN Not. 39, 4 (April 2004), 49-57.
\end{itemize}
Submit your answers to the following questions:\\
 
\paragraph{Question 3A} gprof hands-on\\
In this exercise you will use gprof to examine and improve a piece of code called simpleCode.c. Turn in your answer to question 4.

\begin{enumerate}
\item Look over the source code provided. Build and run the code. Step through the code in the debugger, examining the values of number and searchkey.
\item Evaluate the code using gprof
\item Define and try a few changes to the code to improve the performance
\item Write a short summary of what you learned about the code, what you tried and why, the results,
and whether or not the results matched your expectations.
\end{enumerate}

\paragraph{Question 3B} Questions on the reading:\\
\begin{enumerate}
\item What is the difference between their “dynamic call graph” and “static call graph”? Can you give
an example of something that would not appear in a dynamic call graph but would appear in a static call graph?
\item $[$ old paper alert $]$ What is a “time-sharing” system? We don’t use this term much anymore, but we use time-sharing systems all the time. What is an example of one you use at PSU?
\item How accurate are the $\%$ time amounts reported by gprof?
\item Gprof samples the program counter – what does this contain? What is the key benefit of this method? What is an important downside to this method?
\item $[$From the 2003 update$]$ What was one challenge the authors mention to using the profiler on the operating system kernel?
\end{enumerate}



\paragraph{Question 3c} Showcasing gprof
Develop a code in C or C++ that yields interesting results with gprof. You may use “example code 2” as a starting point. Think about what characteristics in the source code might lead to any of the following: an illustration of a case where gprof results are not quite accurate; an illustration of a difference between a longer function called a few times versus a shorter function called a large number of times; recursion; or any other gprof feature you want to highlight.
Turn in your code plus a brief description of what it shows.




\end{document}

\documentclass[11pt]{article}
\usepackage{amsmath,amssymb,amsthm}
\usepackage{parskip,float}
\usepackage[shortlabels]{enumitem} 
\usepackage{listings}
%\usepackage[margin=1in]{geometry}
\newcommand{\encode}[1]{\langle #1 \rangle}

\let\tab\quad

\title{\bf HW $\#3$ \\[2ex]
\rm\normalsize CS410/510: Introduction to Performance Measurement, Modeling and Analysis \\ Due February 6, 2019}
\date{February 4, 20190}
\author{Jason Graalum}

\begin{document}
\maketitle

\paragraph{Part A: Hands-on Experience with pthreads(Class Partner : AJ Wood)} 
\begin{enumerate}[(1)]
\item Compare the performance of the sequential version to the performance of the pthreads version. What can you conclude about whether or not one is faster? $[$hint: use confidence intervals$]$\\
\item How many runs do you need to measure to be able to specify a reasonable confidence interval at $90\%$ confidence? At $95\%$ confidence?\\
\end{enumerate}

\paragraph{Part B: Using python for statistics} 
\textit{Write python code to solve the following problems.}
\begin{enumerate}[(1)]

\item Comparing 3 servers
\begin{table}[H]
\begin{tabular}{ l  l  l  l  l }
Program  & S1 exec time $($sec$)$ &  S2 exec time $($sec$)$        &      S3 exec time $($sec$)$   &   $\#$ Instructions     \\
1 & 33.4 & 28.8 & 28.3 & $1.45x10^{10}$ \\
2 &19.9&22.1 &25.3 &$7.97x10^{9}$\\
3 &6.5&5.3&4.7&$3.11x10^{9}$\\
4 &84.3&75.8&80.1&$3.77x10^{10}$\\
5 &101.1&99.4&7.2& $4.56x10^{10}$\\
\end{tabular}
\end{table}

\begin{enumerate}[a]
\item Calculate the mean for the 3 different Systems S1, S2, S3\\
\begin{verbatim}
Mean for the 3 different systems S1, S2, S3
S1 Mean =  49.040000
S2 Mean = 46.280000
S3 Mean = 29.120000

\end{verbatim}

\item Calculate the average across the 3 systems of the MIPS rate for each of Programs 1-5 \\
\begin{verbatim}
Average across the 3 systems of the MIPS rate for each of Program
Progam 1 | RunTime Mean: 30.166667(sec)	MIPS Mean: 480.662983(MIPS)
Progam 2 | RunTime Mean: 22.433333(sec)	MIPS Mean: 355.274889(MIPS)
Progam 3 | RunTime Mean: 5.500000(sec)	MIPS Mean: 565.454545(MIPS)
Progam 4 | RunTime Mean: 80.066667(sec)	MIPS Mean: 470.857619(MIPS)
Progam 5 | RunTime Mean: 69.233333(sec)	MIPS Mean: 658.642273(MIPS)

\end{verbatim}
\item Using S3 as the basis system, calculate the average speedup for S1 and S2\\
\begin{verbatim}
Average speedup for S1 and S2 with S3 as baseline
S1 Average Speed up = -68.406593%
S2 Average Speed up = -58.928571%

\end{verbatim}
\item Determine the coefficient of variation of the execution times for each of the 3 systems\\
\begin{verbatim}
Coefficient of variation of the execution times
S1 Coef. of Variation =  0.844275
S2 Coef. of Variation =  0.854986
S3 Coef. of Variation =  1.043166

\end{verbatim}
\end{enumerate}


\item Reporting Meaningful Results\\
We want to determine, on average, how long it takes to write a file of a particular size to a disk drive.\\
We take 8 measurements: 8.0 7.0 5.0 9.0 9.5 11.3 5.2 8.5 \\

\begin{enumerate}[label=\alph*]
\item Calculate a $90\%$ confidence interval for the mean time.\\
\begin{verbatim}
8 sample t-score = 1.895 (from df = 7 and A = 0.05)
90% confidence interval for mean time(c1, c2) = (6.500573, 9.374427)
\end{verbatim}
\item How many measurements would be required to be $90\%$ confident that the mean value is within $7\%$ of the actual value?\\
\begin{verbatim}
Measurements for 90% confidence with 7% error
n = 15
\end{verbatim}
\end{enumerate}

\end{enumerate}

\end{document}
